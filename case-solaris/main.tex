\documentclass[12pt]{article}
\usepackage[utf8]{inputenc}
\usepackage{geometry, amsmath, latexsym, amssymb, graphicx}

\geometry{margin = 1in, headsep = 0.25in}

\begin{document}

\title{Case Solaris}
\maketitle

\textbf{1. Se pudesse ajudar Susan a entender o que houve, como você explicaria?}

O risco da carteira foi calculada erroneamente, visto que não considerou que todas as ações estão se movimentando em conjunto, ou seja, é necessário calcular, primeiramente, a  matriz de covariância das médias dos retornos e depois utilizar a seguinte fórmula para o cálculo do risco do portfólio:

$$
\alpha = \sqrt{wCw^T}
$$

Deixando mais claro, o risco é dado por

$$
\alpha_p^2 = w^T \sum{w}
$$

e \textit{Sigma} aqui é dado pela matriz de covariância

$$
\sum = \left(
$$


\textbf{2. Salve Midas Capital. Ajude Susan e Gauss a compreenderem o acontecido para não cometerem novamente. O que pode ter acontecido?}
\end{document}
