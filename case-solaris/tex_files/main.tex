\documentclass[12pt]{article}
\usepackage[utf8]{inputenc}
\usepackage{geometry, amsmath, latexsym, amssymb, graphicx}

\geometry{margin = 1in, headsep = 0.25in}

\parskip 12pt

\begin{document}

\title{Case Solaris}
\author{Douglas Cardoso}
\maketitle

\textbf{1. Se pudesse ajudar Susan a entender o que houve, como você explicaria?}

O cálculo da volatilidade (risco) da carteira foi feito de forma errada. Em um portfólio, a variação da carteira não é simplesmente a soma de todas as variações das ações subjacentes, visto que temos a correlação entre os ativos, isto é, podem se mover juntos ou em direções opostas, que influencia o risco do portfólio. Assim, o cálculo correto para a volatilidade de um portfólio é:

$$
\sigma_{pf}^2 = w_1^2 \sigma_1^2 + w_2^2 \sigma_2^2 + 2w_1 w_2\sigma_{1,2}$$

Para entender melhor a fórmula, podemos reescrevê-la em formato matricial

\begin{equation*}
\sigma_{pf}^2 = 
\begin{bmatrix}
w_1 & w_2
\end{bmatrix}
\begin{bmatrix}
\sigma_1^2 & \sigma_{1,2}\\
\sigma_{2,1} & \sigma_2^2
\end{bmatrix}
\begin{bmatrix}
w_1\\
w_2
\end{bmatrix}
\end{equation*}

Nessa equação, temos os pesos transpostos multiplicado pela matriz de covariância vezes os pesos dos ativos. A matriz de covariância abriga a variância entre os ativos na diagonal e a covariância entre os ativo 1 e 2 fora da diagonal ($\sigma_{1,2}$). 



\textbf{2. Salve a Midas Capital. Ajude Susan e Gauss a compreenderem o acontecido para não cometerem novamente. O que pode ter acontecido?}

Susan e Gauss estão analisando tempos diferentes, isto é, o \textit{backtest} foi feito em um período de ~6 anos, mas eles estão analisando resultados de curto prazo. A depender do \textit{zoom in} que utilizamos para analisar a curva do gráfico dado, veríamos que no período de semanas ou meses, os ativos variam bastante e nem sempre para cima. Essa assimetria de "janela dos tempo" analisada pode dar a impressão de que a estratégia não está funcionando, mas é necessário analisr o longo prazo. Além disso, para obter um retrato melhor e mais detalhado, seria necessário, dentre outros cálculos:

\begin{itemize}
	\item Medir o \textit{maximum drawdown} e suas durações;
	\item Entender os riscos sistemáticos e os riscos idiossincráticos relativo a cada ativo;
	\item Medir o VaR e o CVaR histórico da estratégia;
	\item Ter um benchmark para comparar os retornos da estratégia a este.
\end{itemize}


\textbf{3. Nova Abordagem.}

\textbf{(a) A contribuição de risco de cada ativo ao portfólio foi calculada corretamente?}

Não. A verdadeira contribuição de risco de cada ativo ao portfólio é 

\begin{center}
\begin{tabular}{ c r }
	      & Risk Contribution\\ 
 \hline
 Ativo\_1 & 0.000994\\
 Ativo\_2 & 0.001142\\ 
 Ativo\_3 & 0.006664\\ 
 Ativo\_4 & 0.008496\\  
 Ativo\_5 & 0.001430\\ 
 Ativo\_6 & 0.001812    
\end{tabular}
\end{center}

\textit{O código está no arquivo .ipynb passado no email.}

\textbf{(b) Qual deve ser portfolio $w$ que faz com que a contribuição de risco seja igual para todos os ativos?}

\begin{center}
\begin{tabular}{ c r }
	      & Optimal Weights\\ 
 \hline
 Ativo\_1 & 0.168625\\
 Ativo\_2 & 0.157328\\ 
 Ativo\_3 & 0.162022\\ 
 Ativo\_4 & 0.171272\\  
 Ativo\_5 & 0.174323\\ 
 Ativo\_6 & 0.166429    
\end{tabular}
\end{center}

\textit{O código está no arquivo .ipynb passado no email.}
\end{document}
